\documentclass[10pt,letterpaper]{article}

% packages
\usepackage{nopageno}
\usepackage[top=1in, bottom=1in, left=1in, right=1in]{geometry}
\usepackage[bookmarks=false,hidelinks]{hyperref}
\usepackage{titlesec}
\usepackage{lastpage}
\usepackage{fancyhdr}
\usepackage[usenames,dvipsnames]{xcolor} % colours
\usepackage[inline]{enumitem} % inline lists
\usepackage[backend=bibtex,style=ieee,sorting=ydnt,doi=false,isbn=false,url=false,defernumbers=true]{biblatex}
\usepackage[romanian]{babel}
\usepackage{combelow}
\usepackage{lipsum}

%-----------------------


\usepackage{xpatch}% or use http://tex.stackexchange.com/a/40705

\newbibmacro*{name:bold}[2]{%
  \def\do##1{\ifstrequal{#1, #2}{##1}{\bfseries\listbreak}{}}%
  \dolistloop{\boldnames}}
\newcommand*{\boldnames}{}

\xpretobibmacro{name:last}{\begingroup\usebibmacro{name:bold}{#1}{#2}}{}{}
\xpretobibmacro{name:first-last}{\begingroup\usebibmacro{name:bold}{#1}{#2}}{}{}
\xpretobibmacro{name:last-first}{\begingroup\usebibmacro{name:bold}{#1}{#2}}{}{}
\xpretobibmacro{name:delim}{\begingroup\normalfont}{}{}

\xapptobibmacro{name:last}{\endgroup}{}{}
\xapptobibmacro{name:first-last}{\endgroup}{}{}
\xapptobibmacro{name:last-first}{\endgroup}{}{}
\xapptobibmacro{name:delim}{\endgroup}{}{}

% just for demonstration
\ExecuteBibliographyOptions{maxnames=99,firstinits}

\forcsvlist{\listadd\boldnames}
  {{Grigore, Elena~Corina}, {Grigore, E.~C.}, {Grigore, Elena\bibnamedelima Corina},
   {Grigore, E\bibinitperiod\bibinitdelim C\bibinitperiod}}

%-----------------------

% Count total number of entries in each refsection
\AtDataInput{%
  \csnumgdef{entrycount:\therefsection}{%
    \csuse{entrycount:\therefsection}+1}}

% Print the labelnumber as the total number of entries in the
% current refsection, minus the actual labelnumber, plus one
\DeclareFieldFormat{labelnumber}{\mkbibdesc{#1}}    
\newrobustcmd*{\mkbibdesc}[1]{%
  \number\numexpr\csuse{entrycount:\therefsection}-1-#1\relax}

\bibliography{CV.bib}

% customizations
%\titleformat{\section}{\Large\uppercase}{\thesection}{1em}{} % modify section format
%\titlespacing{\section}{0em}{1.2em}{1em}

% custom commands
\newcommand{\vx}{\texorpdfstring{$\vert$}{\textbar}~} % vertical bar; added textorpdfstring due to warning
%\newcommand{\bx}{\texorpdfstring{$\bullet$}{.}~} % bullet
\newcommand{\thing}[2]{{#1} \hfill {#2}}
\renewcommand{\labelitemii}{$\circ$} % replace second-level lists (ii) with hollow circle

% page setup
\pagestyle{fancy}
\fancyhf{} % clear all header and footer fields
\fancyfoot[R]{\textcolor{gray}{Page \thepage\ of \pageref{LastPage}}}
\renewcommand{\headrulewidth}{0pt}
\setcounter{secnumdepth}{0} % suppress section numbering
\setlength\parindent{0pt} % no indent

%title
\title{\vspace{-5em}
	\Huge{\textsc{Elena Corina Grigore}}
	\vspace{-0.7em}
	\\\noindent\rule{\linewidth}{0.5pt}
	\vspace{-2.5em}
	\\\noindent\rule{\linewidth}{0.5pt}
	\vspace{-3.2em}
}
%\author{}
\date{}

\begin{document}

\maketitle
\thispagestyle{fancy}

%Yale University, Department of Computer Science \hfill Ph.D. Candidate, Yale University\\
%51 Prospect Street, Office 505 \hfill \href{mailto:elena.corina.grigore@yale.edu}{elena.corina.grigore@yale.edu}\\
%New Haven, CT, 06511 USA \hfill \href{http://elenacorinagrigore.com}{elenacorinagrigore.com}

%nuTonomy c/o Elena Corina Grigore \hfill Research Scientist, nuTonomy\\
%100 Northern Ave, 2nd floor \hfill \href{mailto:elena.corina.grigore@nutonomy.com}{elena.corina.grigore@nutonomy.com}\\
%Boston, MA 02210, USA \hfill \href{http://elenacorinagrigore.com}{elenacorinagrigore.com}

Senior Research Scientist, Motional (formerly nuTonomy) \hfill \href{mailto:elena.corina.grigore@aya.yale.edu}{elena.corina.grigore@aya.yale.edu}\\
Boston, MA, USA \hfill \href{http://elenacorinagrigore.com}{elenacorinagrigore.com}

\vspace{-0.6em}
\section{Research Interests}
\vspace{-0.5em}
Robotics, autonomous driving, machine learning, deep learning, artificial intelligence, human-robot collaboration, adaptive systems, reinforcement learning, multi-agent systems.

\vspace{-0.5em}
\section{Work Experience}

\begin{itemize}%\setlength\itemsep{0em}

%\item \thing{\bf Research Scientist at nuTonomy, Boston, USA}{2018 -- present}\\
\item \thing{\bf Senior Research Scientist at Motional, Boston, MA, USA}{September, 2020 -- present}\\
 	{\bf Research Scientist at Motional, Boston, MA, USA}\hfill{October, 2018  --  September, 2020}\vspace{0.5em}\\
	{\bf Prediction and Behavior Modeling team}\vspace{0.5em}\\
      	%Working on the Prediction and Behavior Modeling team, focusing on road actors present in the environment of the autonomous vehicle (e.g., vehicles, bicyclists, pedestrians, etc.). Applying machine learning techniques (notably, deep learning) to model and predict the trajectories and intentions of such agents.
      	Working on the Prediction and Behavior Modeling team, applying machine learning techniques (notably, deep learning) to model and predict the trajectories and intentions of all the agents of interest in the environment of a self-driving vehicle~\cite{grigore2020covernet}, \cite{grigore2020self_driving_domain_knowledge}.
	
	{\it Outcome:} Utilizing and expanding my research skills to develop models of how agents behave on the road, and to publish state-of-the-art solutions to top machine learning conferences. Applying large-scale deep learning models to real-world, large datasets involving temporal data. Utilizing my knowledge of Python and PyTorch, being part of the full process of creating and deploying deep learning models. Working in a dynamic and fast-paced team with strong collaboration practices, as well as working with other teams to understand data constraints and establish interfacing between modules. Building leadership skills by mentoring interns and supervising their research projects.
	
\item \thing{\bf Research Intern at Uber Advanced Technologies Group}{June, 2017 -- August, 2017}\\
        {\bf San Francisco, CA, USA} \\
        {\bf Deep learning for self-driving car perception team}\vspace{0.5em}\\
        Worked on the perception module of the self-driving pipeline, where the aim was to detect all targets of interest in the environment of the autonomous vehicle. Researched introducing temporal context into deep learning networks, including the use of multi-frames and recurrent neural networks.
	
	{\it Outcome:} Gained experience using large scale deep learning models for detection, and developed research skills relevant to working with region-based convolutional neural networks and recurrent neural networks. Gained experience using the Google Object Detection codebase, TensorFlow, and its associated utilities for working with large datasets (e.g., TFRecords). Worked in a fast-paced team, and collaborated with colleagues to implement novel ideas for the team's deep learning models.

\end{itemize}


\vspace{-1.5em}
\section{Education}
\vspace{-0.5em}
\begin{itemize}%\setlength\itemsep{0em}
\item \thing{\bf Doctor of Philosophy, Computer Science, Yale University, USA}{2018}\\
	Advisor: Brian Scassellati\\
	Ph.D. Thesis: Learning Supportive Behaviors for Adaptive Robots in Human-Robot Collaboration \\
	%\hphantom{Area of study:} in Human-Robot Collaboration \\
	Available at: \href{https://scazlab.yale.edu/sites/default/files/files/corina_dissertation.pdf}{\url{https://scazlab.yale.edu/sites/default/files/files/corina_dissertation.pdf}}
		
\item \thing{\bf Master of Philosophy, Computer Science, Yale University, USA}{2015}
\item \thing{\bf Master of Science, Computer Science, Yale University, USA}{2015}
\item \thing{\bf Master of Engineering with Study Abroad\\Computer Science, University of Bristol, UK}{2012}\\
	Advisors: Kerstin Eder (University of Bristol, UK)\\
	\hphantom{Advisors:} Anthony G. Pipe (Bristol Robotics Laboratory, UK)\\
	\hphantom{Advisors:} Christopher Melhuish (Bristol Robotics Laboratory, UK)\\
	Master's Thesis: {\it ``I Robot, I Think''}\\
	Applied machine learning techniques to model users' intentions for object handovers in human-
robot interaction scenarios~\cite{grigore2013joint}.

	4-year program encompassing my Bachelor's degree\\
	Study Abroad at University of California, San Diego (2010/2011)\\
	Master of Engineering with First Class Honors
\item \thing{\bf Coventry University, UK}{2009}\\
	Completed first year of Computing Honors Degree\\
	Highest scoring student in my cohort\\
	Transfer to University of Bristol at the end of my first undergraduate year
\end{itemize}

%\vspace{-1.7em}
%\section{Recent Work}
%\begin{itemize}%\setlength\itemsep{0em}
%%\item \thing{\bf Research Scientist at nuTonomy, Boston, USA}{2018 -- present}\\
%\item \thing{\bf Senior Research Scientist at Motional, Boston, MA, USA}{2020 -- present}\\
% 	{\bf Research Scientist at Motional, Boston, MA, USA}\hfill{2018  --  \hspace{2.8mm} 2020}\vspace{0.5em}\\
%      	Working on the Prediction and Behavior Modeling team, applying machine learning techniques (notably, deep learning) to model and predict the trajectories and intentions of all the agents of interest in the environment of a self-driving vehicle~\cite{grigore2020covernet}, \cite{grigore2020self_driving_domain_knowledge}
%\item \thing{\bf Research Intern at Uber Advanced Technologies Group, San Francisco, CA, USA}{2017}\\
%      	Integrating temporal context into deep learning networks for self-driving vehicle perception 
%\item \thing{\bf Recent Dissertation Work, Yale University, New Haven, CT, USA}{2017 -- 2018}\\
%	Learning task and user preference models in human-robot collaboration for predicting \\ useful supportive behaviors, tailored to a human peer~\cite{grigore2018pref_assistance}, \cite{grigore2018predict_supportive}
%	%Learning high-level activities during a human-robot collaboration task for predicting features \\ that prompt the robot to provide assistance \cite{grigore2017learn_activities}
%\end{itemize}

\vspace{-1.8em}
\section{Publications}
\nocite{*} % print without citations, * for all
\setlength{\biblabelsep}{0.5em} % separation between number and text
\printbibliography[heading=none,keyword=publication]

%\newpage
\section{Honors and Awards}
\begin{itemize}%\setlength\itemsep{0em}
\item \thing{\bf Best Paper Finalist, Intelligent Virtual Agents (IVA)}{2016}\\
	``Verbal Communication Improves Perceptions of Friendship and Social Presence\\
	 in Human-Robot Interaction''
\item \thing{\bf Best Student Paper Finalist, International Conference on Social Robotics (ICSR)}{2016}\\
	``Comparing Ways to Trigger Migration between a Robot\\
	 and a Virtually Embodied Character''
\item \thing{\bf Human-Robot Interaction (HRI) Pioneer}{2016}\\
	 Highly selective workshop that seeks to foster creativity and collaboration across HRI
\item \thing{\bf Tocher Fellowship, Yale University, USA}{2015}
\item \thing{\bf Tocher Fellowship, Yale University, USA}{2014}
\item \thing{\bf EPSRC (Engineering and Physical Sciences Research Council) Fellowship, UK}{2011}\\
	Summer Research Project at the Bristol Robotics Lab, Bristol, UK
\item \thing{\bf EPSRC Fellowship, UK}{2010}\\
	Summer Research Project at the Bristol Robotics Lab, Bristol, UK
\item \thing{\bf Head of Promotion Honorary Prize (Valedictorian),\\Piatra Neam\cb{t} Computer Science High School, Romania}{2008}
\end{itemize}

\section{Invited Talks}

\begin{itemize}%\setlength\itemsep{0em}

\item \thing{IEEE International Conference on Robotics and Automation (ICRA) \\ Long-term Human Motion Prediction Workshop, Xi'an, China \\ \it Motion Forecasting for Autonomous Driving Applications}{2021}
\item \thing{Virtual Assistant Summit, San Francisco, CA, USA \\ \it Can You Lend Me a Hand? Helpers of the Future}{2017}
\item \thing{STEM Coffee Hour Facilitator, Cheshire, CT, USA \\ \it How is AI Shaping Robotics?}{2017}
\item \thing{IEEE/RSJ International Conference on Intelligent Robots and Systems (IROS) \\ International Workshop on Developmental Social Robotics (DevSor), Tokyo, Japan \\ \it Feasibility of SAR Approaches? Helping Children with Learning Tasks}{2013}

\end{itemize}


\section{Theses}
\printbibliography[heading=none,keyword=phd_thesis,omitnumbers=true,prefixnumbers={Ph.D. Thesis}]
\printbibliography[heading=none,keyword=ma_thesis,omitnumbers=true,prefixnumbers={Master's Thesis}]

%%\vspace{-1.2em}
%\section{Work Experience}
%
%\begin{itemize}%\setlength\itemsep{0em}
%
%%\item \thing{\bf Research Scientist at nuTonomy, Boston, USA}{2018 -- present}\\
%\item \thing{\bf Senior Research Scientist at Motional, Boston, MA, USA}{2020 -- present}\\
% 	{\bf Research Scientist at Motional, Boston, MA, USA}\hfill{2018  --  \hspace{2.8mm} 2020}\vspace{0.5em}\\
%	{\bf Prediction and Behavior Modeling team}\vspace{0.5em}\\
%      	Working on the Prediction and Behavior Modeling team, focusing on the agents of interest present in the self-driving car's environment (e.g., vehicles, bicyclists, pedestrians, etc.). Applying machine learning techniques (notably, deep learning) to model and predict the trajectories and intentions of such agents~\cite{grigore2020covernet}, \cite{grigore2020self_driving_domain_knowledge}.
%	
%	{\it Outcome:} Utilizing and expanding my research skills to develop models of how agents behave on the road, and to publish state-of-the-art solutions to top machine learning conferences. Applying large-scale deep learning models to real-world, large datasets involving temporal data. Utilizing my knowledge of Python and PyTorch, being part of the full process of creating and deploying deep learning models. Working in a dynamic and fast-paced team with strong collaboration practices, as well as working with other teams to understand data constraints and establish interfacing between modules.
%
%\item \thing{\bf Research Intern at Uber Advanced Technologies Group, San Francisco, USA}{2017}\\
%        {\bf Deep learning for self-driving car perception team}\vspace{0.5em}\\
%        Worked on the perception module of the self-driving pipeline, where the aim was to detect all targets of interest in the environment of the autonomous vehicle. Researched introducing temporal context into deep learning networks, including the use of multi-frames and recurrent neural networks.
%	
%	{\it Outcome:} Gained experience using large scale deep learning models for detection, and developed research skills relevant to working with region-based convolutional neural networks and recurrent neural networks. Gained experience using the Google Object Detection codebase, TensorFlow, and its associated utilities for working with large datasets (e.g., TFRecords). Worked in a fast-paced team, and collaborated with colleagues to implement novel ideas for the team's deep learning models.
%		
%\item \thing{\bf Student-teacher at Sidney Stringer School, Coventry, UK}{2009}\\
%        {\bf The Student Associates Scheme, UK}\vspace{0.5em}\\
%	Worked within the Mathematics Department as a student-teacher providing help for students during classes, raising students' aspirations for higher education. Produced and delivered presentations and a programming-based project and also delivered a lesson. 
%	
%	{\it Outcome:} Developed communication, presentation and leadership skills, effectively coordinated groups of students and worked together with teachers and other student-teachers in a motivating environment.
%
%%\item \thing{\bf Course Representative, Coventry University, Coventry, UK}{2008 -- 2009}\\
%%	Speaking on behalf of the students to improve communication between lecturers and students and to solve any course-related problems.
%%	
%%	{\it Outcome:} Persuasion and negotiation skills gained from taking part in meetings with lecturers and staff where representing the students’ point of view.
%
%\end{itemize}

\section{Research Experience}
\begin{itemize}%\setlength\itemsep{1em}
\item {\bf Yale University, Social Robotics Laboratory, New Haven, CT, USA}\\
	\begin{itemize}\setlength\itemsep{0.8em}\vspace{-1em}
	\item \thing{\it Learning Supportive Behaviors for Adaptive Robots \\ in Human-Robot Collaboration}{2014 -- 2018}\vspace{0.1em}\\
	Applying machine learning techniques to endow robots with learning capabilities needed when placed in new environments or faced with new tasks. This includes learning about the structure and progression of a physical task, as well as about the actions human workers perform during this task. Investigating techniques including Hidden Markov Models and reinforcement learning in single- and multi-agent settings, where the robot's aim is to provide supportive behaviors in human-robot collaboration scenarios.
	\item \thing{\it User modeling for motivational states within a reinforcement learning framework}{2013 -- 2015}\vspace{0.1em}\\
	Designed a system for long-term robot companions that employs a model of users' daily motivational states within a reinforcement learning framework.
	\item \thing{\it Developed a robot for interaction with children in an educational setting}{2012 -- 2014}\vspace{0.1em}\\
	 Built, assembled, and programmed research robot platform DragonBot for interaction with children. Performed human-robot interaction study at local schools.
	\end{itemize}
\item {\bf University of Bristol and the Bristol Robotics Laboratory, Bristol, UK}\\
	\begin{itemize}\setlength\itemsep{0.8em}\vspace{-1em}
	\item \thing{\it Master of Engineering ``I Robot, I Think'' Thesis Project}{2011 -- 2012}\vspace{0.1em}\\
	Applied machine learning techniques to model users' intentions for object handovers in human-robot interaction scenarios~\cite{grigore2013joint}.
	\item \thing{\it ``I Robot... I Learn'' Summer Research Project}{2011}\vspace{-0.1em}\\
	Implemented a machine learning algorithm for estimating the state of object handovers in human-robot interaction scenarios.
	\item \thing{\it ``I Robot... and Beyond'' Summer Research Project}{2010}\vspace{-0.1em}\\
	 Investigated safety and liveness properties rooted in design verification principles for a human-robot interaction system~\cite{grigore2011towards}. 
	\end{itemize}
\end{itemize}

\section{Skills}
\begin{itemize}%\setlength\itemsep{0em}
\item Programming languages: Python, R, Matlab, Java, C++, HTML, PHP, CSS, LaTeX
\item Libraries: PyTorch, TensorFlow, NumPy, Brown-UMBC Reinforcement Learning and Planning (BURLAP)
\item Software/IDEs: Git, PyCharm, Eclipse, Visual Studio, NetBeans, Xcode
\item Robotics/hardware platforms: Baxter, Keepon, Nao, ROS, YARP, PhaseSpace Motion Capture System
\end{itemize}

\section{Teaching Experience and Mentorship}
\begin{itemize}\setlength\itemsep{0em}
\item \thing{Supervising research internship projects at Motional, USA}{2018 -- present}
\item \thing{Mentoring high-school and undergraduate students on research projects \\ at Yale University, USA}{2013 -- 2018}
\item Teaching Fellow at Yale University, USA
	\begin{itemize}\setlength\itemsep{0em}
	\item \thing{Natural Language Processing (CPSC 577)}{2017}
	\item \thing{Mathematical Tools for Computer Science (CPSC 202A)}{2014 -- 2015}
	\item \thing{Intelligent Robotics (CPSC 473)}{2013 -- 2015}
	\item \thing{Intelligent Robotics Lab (CPSC 472)}{2013}
	\end{itemize}
\item \thing{Point of contact for incoming Romanian students, University of Bristol, UK}{2009 -- 2012}
\item \thing{Mathematics student-teacher at Sydney Stringer School, Coventry, UK\\Students Associates Scheme}{2009}
\item \thing{Course Representative, Coventry University, Coventry, UK\\Speaking on behalf of the student body}{2008 -- 2009}
\end{itemize}

\section{Academic Service and Membership}
\begin{itemize}%\setlength\itemsep{0em}
\item Conference and Workshop Committee Leadership
	\begin{itemize}\setlength\itemsep{0em}
	\item \thing{AAAI Conference on Artificial Intelligence (AAAI) \\Program Committee Member}{2020 -- 2021}
	\item \thing{International Conference on Intelligent Virtual Agents (IVA) \\Program Committee Member}{2017 -- 2020}
	\item \thing{ACM/IEEE International Conference on Human-Robot Interaction (HRI) \\Program Committee Member and Pioneers Workshop Panel Chair}{2017}
	\item \thing{IEEE/RSJ International Conference on Intelligent Robots and Systems (IROS) \\Synergies Between Learning and Interaction (SBLI) Workshop\\Program Committee Member}{2017}
	\end{itemize}
\item Conference Refereeing service
	\begin{itemize}\setlength\itemsep{0em}
	\item \thing{International Conference on Machine Learning (ICML)}{2019}
	\item \thing{ACM/IEEE International Conference on Human-Robot Interaction (HRI)}{2015 -- 2019}
	\item \thing{IEEE/RSJ International Conference on Robotics and Automation (ICRA)}{2018}
	\item \thing{Springer International Journal of Social Robotics}{2018}
	\item \thing{IEEE/RSJ International Conference on \\ Intelligent Robots and Systems (IROS)}{2014, 2017 -- 2018}
	\item \thing{ACM/IEEE Interaction Design and Children Conference (IDC)}{2018}
	\item \thing{EEE-RAS International Conference on Humanoid Robots (Humanoids)}{2017}
	\item \thing{IEEE Transactions on Automation Science and Engineering}{2017}
	\item \thing{International Conference on Social Robotics (ICSR)}{2016}
	\item \thing{IEEE International Symposium on \\ Robot and Human Interactive Communication (RO-MAN)}{2016}
	\item \thing{Elsevier Cognitive Systems Research Journal}{2016}
	\item \thing{Affective Computing and Intelligent Interaction}{2015}
	\end{itemize}
\item Membership in Professional Societies
	\begin{itemize}\setlength\itemsep{0em}
	\item\thing{ Association for the Advancement of Artificial Intelligence}{2014 -- present}
	\item \thing{IEEE}{2014 -- present}
	\item \thing{Cognitive Science Society}{2014 -- present}
	\end{itemize}
\item Outreach
	\begin{itemize}
	\item \thing{World Science Festival, New York City}{2014}
	\item \thing{Routine lab tours and open houses, Yale Social Robotics Lab, CT}{2012 -- 2017}
	\item \thing{Routine outreach activities involving robot demos at local schools, CT}{2012 -- 2017}
	\end{itemize}
\item Book Reviewing
	\begin{itemize}
	\item \thing{{\it Visual Analysis of Behaviour -- From Pixels to Semantics}, by Gong S, Xiang T}{2012}
	\end{itemize}
\end{itemize}
               
\section{Conferences and Summer Schools Attended}
\begin{itemize}%\setlength\itemsep{0em}
\item \thing{IEEE/CVF Conference on Computer Vision and Pattern Recognition (CVPR)\\Attended conference for accepted paper}{2020}
\item \thing{IEEE/RSJ International Conference on Intelligent Robots and Systems (IROS)\\Presented talk for accepted paper}{2018}\\
         Presented talk for accepted paper and invited talk for the DevSor Workshop \rlap{\hspace*{30.5mm}2013}
\item \thing{International Conference on Autonomous Agents and Multiagent Systems (AAMAS)\\Presented talk for accepted papers}{2018}
\item \thing{Robotics: Science and Systems\\Presented talk for accepted paper (RSS)}{2017}
\item \thing{ACM/IEEE International Conference on Human-Robot Interaction (HRI)\\Organized and moderated the Pioneers Workshop Panel }{2016}\\
        Presented talk for accepted paper \rlap{\hspace*{96.4mm}2015}
\item \thing{Annual Conference on Neural Information Processing Systems (NeurIPS)\\Presented talk for workshop full paper}{2016}
\item \thing{International Conference on Intelligent Virtual Agents (IVA)\\Presented paper for best paper finalist category}{2016}
\item \thing{International Conference on Social Robotics (ICSR)\\Presented paper for best student paper finalist category}{2016}
\item \thing{International Conference on Machine Learning (ICML)}{2016}
\item \thing{International Joint Conference on Artificial Intelligence (IJCAI)}{2016}
\item \thing{AAAI Fall Symposium Series\\Presented talk for accepted paper}{2015}
\item \thing{Max Planck Institute for Intelligent Systems Machine Learning Summer School, Germany\\({\bf 20\% acceptance rate})}{2015}
\item \thing{International Conference on User Modelling, Adaptation and Personalization (UMAP)\\Presented talk for accepted paper}{2015}
%\item \thing{ACM/IEEE International Conference on Human-Robot Interaction (HRI)\\Presented talk for accepted paper}{2015}
\item \thing{AAAI Conference on Artificial Intelligence (AAAI)\\Presented robot demo}{2014}
\item \thing{Cognitive Science Society Annual Conference (CogSci)\\Presented robot demo}{2014}
%\item \thing{IEEE/RSJ International Conference on Intelligent Robots and Systems (IROS)\\Presented talk for accepted paper and invited talk for the DevSor Workshop}{2013}
\item \thing{The First Summer School on Social Human-Robot Interaction, UK}{2013}
\item \thing{Conference Towards Autonomous Robotic Systems (TAROS)\\Presented talk for accepted paper}{2011}
\end{itemize}

\section{Languages}
\begin{itemize}%\setlength\itemsep{0em}
\item Romanian -- native language
\item English -- fluent: written and spoken
\item Spanish -- conversational: spoken
\item French -- basic: written and spoken
\end{itemize}

\end{document}